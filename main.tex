\documentclass[11pt, twocolumn]{article}
% Paquetes adicionales
\usepackage[utf8]{inputenc} % Codificación de caracteres
\usepackage{amsmath} % Paquete para matemáticas
\usepackage{graphicx} % Para incluir gráficos
\usepackage{hyperref} % Para hipervínculos
\usepackage[spanish]{babel}
\usepackage{tabularx} % Paquete para tablas ajustables
\usepackage{float} % Paquete para la opción H en tablas y figuras
\usepackage[letterpaper,top=2cm,bottom=2cm,left=3cm,right=3cm,marginparwidth=1.75cm]{geometry}
\usepackage{algorithm} % Paquete para algoritmos
\usepackage{algpseudocode} % Paquete para pseudocódigo
\usepackage{titlesec} % Paquete para ajustar espacios de secciones
% Configuración de márgenes
\geometry{margin=1in}
% Ajuste del espacio entre columnas
\setlength{\columnsep}{20pt} % Cambia el valor según sea necesario
% Ajuste del espacio entre columnas
\setlength{\columnsep}{20pt} % Cambia el valor según sea necesario
\titlespacing*{\subsubsection}{0pt}{\baselineskip}{0.2\baselineskip}
%Documento
\begin{document}
% Título y autores
\twocolumn[
    \begin{@twocolumnfalse}
        \title{Algoritmo Híbrido para la generación de un Ultimate Team en FIFA 19}
        \author{
            20210836\\
            \textit{Piero Marcelo Pastor Pacheco}\\
            \and
            20200445\\
            \textit{Aarón Ulises Santillán Huamán}
        }
        \date{\today}
        \maketitle
        \begin{abstract}
                En este artículo se compararán un algoritmo metaheurístico híbrido con otro que sea puro, esto para saber si los costes computacionales que puedan exigirnos el primero mencionado valdrían la pena o no. Además todo esto se verá implementado para la generación de un equipo del videojuego de deportes FIFA 19, buscando siempre dar la solución que gaste menos dinero ofreciendo un equipo aceptable. Todo esto siguiendo los lineamientos que requieren los problemas de asignación.\newline
        \end{abstract}
    \end{@twocolumnfalse}
]

\section{Introdución}

Normalmente para la elaboración de un equipo para algún deporte, existe siempre la limitante del presupuesto. El algoritmo que se presentará a continuación trata de bajo un presupuesto por más mínimo que sea, conseguir generar un buen equipo de fútbol. Esto se llevará a cabo maximizando la media en las posiciones de jugadores, potencial de juego, y química; mientras que se buscará minimizar el precio para siempre ahorrar lo mayor posible, y la edad de los jugadores teniendo preferencia por los más jovenes.\newline
Además de solucionar este problema, se buscará comparar un algoritmo híbrido que sea la combinación de un genético evolutivo y un GRASP (Greedy Randomized Adapatative Search Procedure), con un algoritmo genético evolutivo puro.\newline
Los resultados nos ofrecerán una comparación para saber si vale la pena el costo computacional extra de un GRASP dentro de un evolutivo, o si la diferencia de resultados es mínima y no es necesario; considerándose así un desperdicio la implementación del híbrido.\newline

\section{Trabajos relacionados}
\subsection{Publicacion 1}
\subsection{Publicacion 2}
\subsection{Publicacion 3}

\section{Metodología}
\subsection{Descripción del problema}
El artículo busca solucionar el problema de la generación de equipos de fútbol, pero al mismo tiempo busca realizar un análisis a los resultados de dos tipos de algoritmos metaheurísticos.\newline
El algoritmo híbrido buscará mejorar mediante la selección, la media y química del equipo; minimizando gastos para que así el usuario pueda seleccionar más adelante a un jugador de élite de su gusto (en caso el algoritmo no lo haya colocado ya en su equipo). El GRASP se verá reflejado en la generación de una población inicial, que si bien puede ser randonomica, esta siempre buscará maximizar el fitness tanto del jugador individual como del equipo general. El resto del algoritmo evolutivo mantiene los estándares comunes, implementando una función que elimina las aberraciones, otra que se encarga de los clones, y una última que con cada generación limita la población disminuyéndola en un porcentaje; si bien esta última selección es randonomica, siempre da preferencia a eliminar a la población con peor fitness. \newline
El algoritmo evolutivo puro, trabaja de la misma manera que el híbrido en el ámbito que corresponde a un algoritmo genético. La única diferencia es la forma en que se genera la población inicial, este algoritmo la genera randonomicamente, solo tiene como restricción el no crear aberraciones.

\subsection{Entrada y Salida}
Ambos algoritmos recibirán un grupo de jugadores del juego, aquí se encontrará por jugador los siguientes valores:

\begin{table}[H]
    \centering
    \begin{tabularx}{\columnwidth}{|X|X|X|}
        \hline
        Atributo & Tipo de dato & Descripción \\
        \hline
        id & Integer & ID del jugador\\
        nombre & String & Nombre del jugador \\
        nacionalidad & String & Nacionalidad del jugador\\
        posicion & String & Posición principal del jugador\\
        media & Integer & Media general del jugador\\
        potencial & Integer & Potencial de juego del jugador\\
        club & String & Club del jugador\\
        valor & Double & Valor en M€\\
        mediasPos & Map<String, Integer> & Media de juego en cada posición del campo\\
        
        \hline
    \end{tabularx}
    \caption{Atributos de un jugador.}
    \label{tab:etiqueta}
\end{table}

Se utilizaron varias posiciones, para que si una posición ya está tomada, pero si este jugador se desenvuelve igual de bien en otra; pueda ser enviado a esa.\newline
La entrada por parte del usuario es únicamente que formación se utilizará y que presupuesto es el que posee para poder armar su equipo.\newline
La salida es el equipo formado en cada posición y el fitness que generó este individuo en la población.

\subsection{Algoritmos empleados}
A continuación se tendrán los pseudocódigos de cada función empleada para el algoritmo, además de la programación metódica que demuestra el correcto funcionamiento de este.

\subsubsection{Metaheurístico Híbrido}

\floatname{algorithm}{Metaheurístico Híbrido}
\renewcommand{\thealgorithm}{}
\begin{algorithm}[H]
    \caption{Algoritmo para encontrar el mejor equipo en el videojuego FIFA 19}\label{euclid}
    \begin{algorithmic}[1]
        \Procedure{hibrido}{$jugadores, b$} \Comment{Encuentra el MCD de $a$ y $b$ usando el algoritmo de Euclides}
            \State \Call{GRASP}{$jugadores, poblacion, n, presupuesto, posiciones, quimica, POB_INICIAL$}
            \While{$r \neq 0$}
                \State $a \gets b$
                \State $b \gets r$
                \State $r \gets a \mod b$
            \EndWhile
            \State \textbf{return} $b$ \Comment{El MCD es $b$}
        \EndProcedure
    \end{algorithmic}
\end{algorithm}

\subsubsection{GRASP}

\subsubsection{Genético Evolutivo}

\section{Experimentación y Resultados}

\section{Conclusión}

\section{Sugerencia de trabajos futuros}

\section{Implicancias éticas}

\section{Repositorio de GitHub}

\section{Declaración de contribución de cada integrante}

\section{Bibliografía}
%\subsection{How to create Sections and Subsections} Para generar una nueva subseccion

Simply use the section and subsection commands, as in this example document! With Overleaf, all the formatting and numbering is handled automatically according to the template you've chosen. If you're using the Visual Editor, you can also create new section and subsections via the buttons in the editor toolbar.

\subsection{How to include Figures}

First you have to upload the image file from your computer using the upload link in the file-tree menu. Then use the includegraphics command to include it in your document. Use the figure environment and the caption command to add a number and a caption to your figure. See the code for Figure \ref{fig:frog} in this section for an example.

Note that your figure will automatically be placed in the most appropriate place for it, given the surrounding text and taking into account other figures or tables that may be close by. You can find out more about adding images to your documents in this help article on \href{https://www.overleaf.com/learn/how-to/Including_images_on_Overleaf}{including images on Overleaf}.
%\href{https://www.overleaf.com/learn/how-to/Including_images_on_Overleaf} AGREGAR HIPERVINCULOS
%\begin{figure} AGREGAR IMAGENES
%\centering
%\includegraphics[width=0.25\linewidth]{frog.jpg}
%\caption{\label{fig:frog}This frog was uploaded via the file-tree menu.}
%\end{figure}

\subsection{How to add Tables}

Use the table and tabular environments for basic tables --- see Table~\ref{tab:widgets}, for example. For more information, please see this help article on \href{https://www.overleaf.com/learn/latex/tables}{tables}. 

\begin{table}
\centering
\begin{tabular}{l|r}
Item & Quantity \\\hline
Widgets & 42 \\
Gadgets & 13
\end{tabular}
\caption{\label{tab:widgets}An example table.}
\end{table}

\subsection{How to add Comments and Track Changes}

Comments can be added to your project by highlighting some text and clicking ``Add comment'' in the top right of the editor pane. To view existing comments, click on the Review menu in the toolbar above. To reply to a comment, click on the Reply button in the lower right corner of the comment. You can close the Review pane by clicking its name on the toolbar when you're done reviewing for the time being.

Track changes are available on all our \href{https://www.overleaf.com/user/subscription/plans}{premium plans}, and can be toggled on or off using the option at the top of the Review pane. Track changes allow you to keep track of every change made to the document, along with the person making the change. 

\subsection{How to add Lists}

You can make lists with automatic numbering \dots

\begin{enumerate}
\item Like this,
\item and like this.
\end{enumerate}
\dots or bullet points \dots
\begin{itemize}
\item Like this,
\item and like this.
\end{itemize}

\subsection{How to write Mathematics}

\LaTeX{} is great at typesetting mathematics. Let $X_1, X_2, \ldots, X_n$ be a sequence of independent and identically distributed random variables with $\text{E}[X_i] = \mu$ and $\text{Var}[X_i] = \sigma^2 < \infty$, and let
\[S_n = \frac{X_1 + X_2 + \cdots + X_n}{n}
      = \frac{1}{n}\sum_{i}^{n} X_i\]
denote their mean. Then as $n$ approaches infinity, the random variables $\sqrt{n}(S_n - \mu)$ converge in distribution to a normal $\mathcal{N}(0, \sigma^2)$.


\subsection{How to change the margins and paper size}

Usually the template you're using will have the page margins and paper size set correctly for that use-case. For example, if you're using a journal article template provided by the journal publisher, that template will be formatted according to their requirements. In these cases, it's best not to alter the margins directly.

If however you're using a more general template, such as this one, and would like to alter the margins, a common way to do so is via the geometry package. You can find the geometry package loaded in the preamble at the top of this example file, and if you'd like to learn more about how to adjust the settings, please visit this help article on \href{https://www.overleaf.com/learn/latex/page_size_and_margins}{page size and margins}.

\subsection{How to change the document language and spell check settings}

Overleaf supports many different languages, including multiple different languages within one document. 

To configure the document language, simply edit the option provided to the babel package in the preamble at the top of this example project. To learn more about the different options, please visit this help article on \href{https://www.overleaf.com/learn/latex/International_language_support}{international language support}.

To change the spell check language, simply open the Overleaf menu at the top left of the editor window, scroll down to the spell check setting, and adjust accordingly.

\subsection{How to add Citations and a References List}

You can simply upload a \verb|.bib| file containing your BibTeX entries, created with a tool such as JabRef. You can then cite entries from it, like this: \cite{greenwade93}. Just remember to specify a bibliography style, as well as the filename of the \verb|.bib|. You can find a \href{https://www.overleaf.com/help/97-how-to-include-a-bibliography-using-bibtex}{video tutorial here} to learn more about BibTeX.

If you have an \href{https://www.overleaf.com/user/subscription/plans}{upgraded account}, you can also import your Mendeley or Zotero library directly as a \verb|.bib| file, via the upload menu in the file-tree.

\subsection{Good luck!}

We hope you find Overleaf useful, and do take a look at our \href{https://www.overleaf.com/learn}{help library} for more tutorials and user guides! Please also let us know if you have any feedback using the Contact Us link at the bottom of the Overleaf menu --- or use the contact form at \url{https://www.overleaf.com/contact}.

\bibliographystyle{alpha}
\bibliography{sample}

\end{document}