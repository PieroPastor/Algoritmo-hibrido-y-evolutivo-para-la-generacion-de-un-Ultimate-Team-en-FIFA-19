\documentclass{article}
% Paquetes adicionales
\usepackage[utf8]{inputenc} % Codificación de caracteres
\usepackage{amsmath} % Paquete para matemáticas
\usepackage{graphicx} % Para incluir gráficos
\usepackage{hyperref} % Para hipervínculos
\usepackage[spanish]{babel}
\usepackage{tabularx} % Paquete para tablas ajustables
\usepackage{float} % Paquete para la opción H en tablas y figuras
\usepackage[a4,top=2cm,bottom=2cm,left=1cm,right=1cm]{geometry}
\usepackage{algorithm} % Paquete para algoritmos
\usepackage{algpseudocode} % Paquete para pseudocódigo
\usepackage{titlesec} % Paquete para ajustar espacios de secciones
% Configuración de márgenes

% Ajuste del espacio entre columnas
% Ajuste del espacio entre columnas
\titlespacing*{\subsubsection}{0pt}{\baselineskip}{0.2\baselineskip}
%Documento
\begin{document}
\floatname{algorithm}{Metaheurístico GRASP}
\renewcommand{\thealgorithm}{}
\begin{algorithm}
    \caption{Algoritmo para elaborar la población inicial del híbrido}
    \begin{algorithmic}[1]
        \Procedure{GRASP}{$jugadores, poblacion, n, presupuesto, posiciones, quimica, requerido$} \Comment{Halla una población inicial aceptable}
            \State $i \gets 0$
            \While{$i \neq ITERACIONES$}
                \State $foparcial \gets$ \Call{Construccion}{$mediasPosicion, pobParcial, n, presupuesto, posiciones, quimica$}
                \State $foparcial \gets foparcial * $ \Call{HallarQuimica}{$pobParcial, jugadores, quimica$} \Comment{Halla la química de todo el equipo, ya que, la construcción no puede hacerlo por si sola}
                \State \Call{ActualizarMejores}{$poblacion, pobParcial, mejoresFo, foParcial, requerido$} \Comment{Actualiza la población inicial con el nuevo individuo conseguido}
                \State $i \gets i + 1$
            \EndWhile
        \EndProcedure
    \end{algorithmic}
\end{algorithm}
\end{document}