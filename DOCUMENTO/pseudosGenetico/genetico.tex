\documentclass{article}
% Paquetes adicionales
\usepackage[utf8]{inputenc} % Codificación de caracteres
\usepackage{amsmath} % Paquete para matemáticas
\usepackage{graphicx} % Para incluir gráficos
\usepackage{hyperref} % Para hipervínculos
\usepackage[spanish]{babel}
\usepackage{tabularx} % Paquete para tablas ajustables
\usepackage{float} % Paquete para la opción H en tablas y figuras
\usepackage[a4,top=2cm,bottom=2cm,left=1cm,right=1cm]{geometry}
\usepackage{algorithm} % Paquete para algoritmos
\usepackage{algpseudocode} % Paquete para pseudocódigo
\usepackage{titlesec} % Paquete para ajustar espacios de secciones
% Configuración de márgenes
% Ajuste del espacio entre columnas
% Ajuste del espacio entre columnas
\titlespacing*{\subsubsection}{0pt}{\baselineskip}{0.2\baselineskip}
%Documento
\begin{document}
\floatname{algorithm}{Genético Evolutivo}
\renewcommand{\thealgorithm}{}
\begin{algorithm}
    \caption{Algoritmo para encontrar el mejor equipo en el videojuego FIFA 19} 
    \begin{algorithmic}[1]
        \Procedure{hibrido}{$jugadores, n, presupuesto, posiciones, quimica$} \Comment{Halla el mejor equipo en base al presupuesto}
            \State \Call{GenerarPoblacion}{$jugadores, poblacion, n, presupuesto, posiciones, quimica, POBINICIAL$}
            \State $i \gets 0$
            \While{$i \neq GENERACIONES \land tampoblacion > 1 $}
                \State \Call{Seleccion}{$poblacion, padres, jugadores, n, posiciones, quimica$}
                \State \Call{Casamiento}{$poblacion, padres$}
                \State \Call{Mutacion}{$poblacion, padres, jugadores$}
                \State \Call{EliminarAberraciones}{$poblacion, jugadores, presupuesto$}
                \State \Call{EliminarClones}{$poblacion$}
                \State \Call{DisminuirPoblacion}{$poblacion, jugadores, n, posiciones, quimica$}
                \State $i \gets i+1$
            \EndWhile
            \State \Call{Seleccion}{$poblacion, padres, jugadores, n, posiciones, quimica$}
            \State \textbf{return } \Call{Random}{$padres$} \Comment{Retorna el mejor equipo}
        \EndProcedure
    \end{algorithmic}
\end{algorithm}
\end{document}